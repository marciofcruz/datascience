% Options for packages loaded elsewhere
\PassOptionsToPackage{unicode}{hyperref}
\PassOptionsToPackage{hyphens}{url}
%
\documentclass[
]{article}
\usepackage{amsmath,amssymb}
\usepackage{lmodern}
\usepackage{ifxetex,ifluatex}
\ifnum 0\ifxetex 1\fi\ifluatex 1\fi=0 % if pdftex
  \usepackage[T1]{fontenc}
  \usepackage[utf8]{inputenc}
  \usepackage{textcomp} % provide euro and other symbols
\else % if luatex or xetex
  \usepackage{unicode-math}
  \defaultfontfeatures{Scale=MatchLowercase}
  \defaultfontfeatures[\rmfamily]{Ligatures=TeX,Scale=1}
\fi
% Use upquote if available, for straight quotes in verbatim environments
\IfFileExists{upquote.sty}{\usepackage{upquote}}{}
\IfFileExists{microtype.sty}{% use microtype if available
  \usepackage[]{microtype}
  \UseMicrotypeSet[protrusion]{basicmath} % disable protrusion for tt fonts
}{}
\makeatletter
\@ifundefined{KOMAClassName}{% if non-KOMA class
  \IfFileExists{parskip.sty}{%
    \usepackage{parskip}
  }{% else
    \setlength{\parindent}{0pt}
    \setlength{\parskip}{6pt plus 2pt minus 1pt}}
}{% if KOMA class
  \KOMAoptions{parskip=half}}
\makeatother
\usepackage{xcolor}
\IfFileExists{xurl.sty}{\usepackage{xurl}}{} % add URL line breaks if available
\IfFileExists{bookmark.sty}{\usepackage{bookmark}}{\usepackage{hyperref}}
\hypersetup{
  pdftitle={Case Salario Candidato},
  pdfauthor={Marcio Cruz},
  hidelinks,
  pdfcreator={LaTeX via pandoc}}
\urlstyle{same} % disable monospaced font for URLs
\usepackage[margin=1in]{geometry}
\usepackage{color}
\usepackage{fancyvrb}
\newcommand{\VerbBar}{|}
\newcommand{\VERB}{\Verb[commandchars=\\\{\}]}
\DefineVerbatimEnvironment{Highlighting}{Verbatim}{commandchars=\\\{\}}
% Add ',fontsize=\small' for more characters per line
\usepackage{framed}
\definecolor{shadecolor}{RGB}{248,248,248}
\newenvironment{Shaded}{\begin{snugshade}}{\end{snugshade}}
\newcommand{\AlertTok}[1]{\textcolor[rgb]{0.94,0.16,0.16}{#1}}
\newcommand{\AnnotationTok}[1]{\textcolor[rgb]{0.56,0.35,0.01}{\textbf{\textit{#1}}}}
\newcommand{\AttributeTok}[1]{\textcolor[rgb]{0.77,0.63,0.00}{#1}}
\newcommand{\BaseNTok}[1]{\textcolor[rgb]{0.00,0.00,0.81}{#1}}
\newcommand{\BuiltInTok}[1]{#1}
\newcommand{\CharTok}[1]{\textcolor[rgb]{0.31,0.60,0.02}{#1}}
\newcommand{\CommentTok}[1]{\textcolor[rgb]{0.56,0.35,0.01}{\textit{#1}}}
\newcommand{\CommentVarTok}[1]{\textcolor[rgb]{0.56,0.35,0.01}{\textbf{\textit{#1}}}}
\newcommand{\ConstantTok}[1]{\textcolor[rgb]{0.00,0.00,0.00}{#1}}
\newcommand{\ControlFlowTok}[1]{\textcolor[rgb]{0.13,0.29,0.53}{\textbf{#1}}}
\newcommand{\DataTypeTok}[1]{\textcolor[rgb]{0.13,0.29,0.53}{#1}}
\newcommand{\DecValTok}[1]{\textcolor[rgb]{0.00,0.00,0.81}{#1}}
\newcommand{\DocumentationTok}[1]{\textcolor[rgb]{0.56,0.35,0.01}{\textbf{\textit{#1}}}}
\newcommand{\ErrorTok}[1]{\textcolor[rgb]{0.64,0.00,0.00}{\textbf{#1}}}
\newcommand{\ExtensionTok}[1]{#1}
\newcommand{\FloatTok}[1]{\textcolor[rgb]{0.00,0.00,0.81}{#1}}
\newcommand{\FunctionTok}[1]{\textcolor[rgb]{0.00,0.00,0.00}{#1}}
\newcommand{\ImportTok}[1]{#1}
\newcommand{\InformationTok}[1]{\textcolor[rgb]{0.56,0.35,0.01}{\textbf{\textit{#1}}}}
\newcommand{\KeywordTok}[1]{\textcolor[rgb]{0.13,0.29,0.53}{\textbf{#1}}}
\newcommand{\NormalTok}[1]{#1}
\newcommand{\OperatorTok}[1]{\textcolor[rgb]{0.81,0.36,0.00}{\textbf{#1}}}
\newcommand{\OtherTok}[1]{\textcolor[rgb]{0.56,0.35,0.01}{#1}}
\newcommand{\PreprocessorTok}[1]{\textcolor[rgb]{0.56,0.35,0.01}{\textit{#1}}}
\newcommand{\RegionMarkerTok}[1]{#1}
\newcommand{\SpecialCharTok}[1]{\textcolor[rgb]{0.00,0.00,0.00}{#1}}
\newcommand{\SpecialStringTok}[1]{\textcolor[rgb]{0.31,0.60,0.02}{#1}}
\newcommand{\StringTok}[1]{\textcolor[rgb]{0.31,0.60,0.02}{#1}}
\newcommand{\VariableTok}[1]{\textcolor[rgb]{0.00,0.00,0.00}{#1}}
\newcommand{\VerbatimStringTok}[1]{\textcolor[rgb]{0.31,0.60,0.02}{#1}}
\newcommand{\WarningTok}[1]{\textcolor[rgb]{0.56,0.35,0.01}{\textbf{\textit{#1}}}}
\usepackage{graphicx}
\makeatletter
\def\maxwidth{\ifdim\Gin@nat@width>\linewidth\linewidth\else\Gin@nat@width\fi}
\def\maxheight{\ifdim\Gin@nat@height>\textheight\textheight\else\Gin@nat@height\fi}
\makeatother
% Scale images if necessary, so that they will not overflow the page
% margins by default, and it is still possible to overwrite the defaults
% using explicit options in \includegraphics[width, height, ...]{}
\setkeys{Gin}{width=\maxwidth,height=\maxheight,keepaspectratio}
% Set default figure placement to htbp
\makeatletter
\def\fps@figure{htbp}
\makeatother
\setlength{\emergencystretch}{3em} % prevent overfull lines
\providecommand{\tightlist}{%
  \setlength{\itemsep}{0pt}\setlength{\parskip}{0pt}}
\setcounter{secnumdepth}{-\maxdimen} % remove section numbering
\ifluatex
  \usepackage{selnolig}  % disable illegal ligatures
\fi

\title{Case Salario Candidato}
\author{Marcio Cruz}
\date{28/06/2021}

\begin{document}
\maketitle

\hypertarget{case}{%
\subsection{Case}\label{case}}

Um recrutador deseja estimar o salário de um candidato, a partir da nota
média de várias provas realizadas durante o processo seletivo de
admissão na empresa. O objetivo é ajudar os gestores a atribuir o
salário do candidato dentro do intervalo já estipulado pela política de
remuneração da empresa.

\hypertarget{hipuxf3teses-para-ux3b21}{%
\subsubsection{Hipóteses para β1:}\label{hipuxf3teses-para-ux3b21}}

\hypertarget{ho-ux3b21-0-nuxe3o-existe-relauxe7uxe3o-linear-entre-as-variuxe1veis}{%
\paragraph{Ho: β1 = 0 (não existe relação linear entre as
variáveis)}\label{ho-ux3b21-0-nuxe3o-existe-relauxe7uxe3o-linear-entre-as-variuxe1veis}}

\hypertarget{ha-ux3b21-0-existe-relauxe7uxe3o-linear-entre-as-variuxe1veis}{%
\paragraph{Ha: β1 \# 0 (existe relação linear entre as
variáveis)}\label{ha-ux3b21-0-existe-relauxe7uxe3o-linear-entre-as-variuxe1veis}}

\hypertarget{hipuxf3teses-para-ux3b20}{%
\subsubsection{Hipóteses para β0:}\label{hipuxf3teses-para-ux3b20}}

\hypertarget{ho-ux3b20-0-passa-pela-origem}{%
\paragraph{Ho: β0 = 0 (Passa pela
origem)}\label{ho-ux3b20-0-passa-pela-origem}}

\hypertarget{ha-ux3b20-0-nuxe3o-passa-pela-origem}{%
\paragraph{Ha: β0 \# 0 (Não passa pela
origem)}\label{ha-ux3b20-0-nuxe3o-passa-pela-origem}}

\hypertarget{respostas}{%
\subsection{Respostas}\label{respostas}}

\hypertarget{a-obtenha-o-gruxe1fico-de-dispersuxe3o-entre-as-variuxe1veis}{%
\subsubsection{a) Obtenha o gráfico de dispersão entre as
variáveis}\label{a-obtenha-o-gruxe1fico-de-dispersuxe3o-entre-as-variuxe1veis}}

\includegraphics{exercicio---Regressão-Simples---Salario-Candidato_files/figure-latex/unnamed-chunk-2-1.pdf}

\hypertarget{b-calcule-o-coeficiente-de-correlauxe7uxe3o-entre-as-variuxe1veis.-uxe9-uma-correlauxe7uxe3o-positiva-ou-negativa-uxe9-uma-correlauxe7uxe3o-forte}{%
\subsubsection{b) Calcule o coeficiente de correlação entre as
variáveis. É uma correlação positiva ou negativa? É uma correlação
forte?}\label{b-calcule-o-coeficiente-de-correlauxe7uxe3o-entre-as-variuxe1veis.-uxe9-uma-correlauxe7uxe3o-positiva-ou-negativa-uxe9-uma-correlauxe7uxe3o-forte}}

\textbf{Interpretação:} Existe uma correlação forte e positiva (r =
0.7765) entre as variáveis de salário e nota média.

\hypertarget{c-obtenha-o-modelo-de-regressuxe3o-linear-simples.-com-90-de-confianuxe7a-huxe1-relauxe7uxe3o-linear-entre-as-variuxe1veis}{%
\subsubsection{c) Obtenha o modelo de regressão linear simples. Com 90\%
de confiança, há relação linear entre as
variáveis?}\label{c-obtenha-o-modelo-de-regressuxe3o-linear-simples.-com-90-de-confianuxe7a-huxe1-relauxe7uxe3o-linear-entre-as-variuxe1veis}}

\begin{Shaded}
\begin{Highlighting}[]
\NormalTok{regressao }\OtherTok{\textless{}{-}} \FunctionTok{lm}\NormalTok{(}\AttributeTok{data=}\NormalTok{salario, salario}\SpecialCharTok{$}\StringTok{\textasciigrave{}}\AttributeTok{Salario Mensal}\StringTok{\textasciigrave{}} \SpecialCharTok{\textasciitilde{}}\NormalTok{ salario}\SpecialCharTok{$}\StringTok{\textasciigrave{}}\AttributeTok{Nota Média}\StringTok{\textasciigrave{}}\NormalTok{)}
\CommentTok{\# summary(regressao)}
\end{Highlighting}
\end{Shaded}

\hypertarget{o-pvalor-0.0000000000344-010-indica-que-devemos-rejeitar-a-hipuxf3tese-nula-ou-seja-aceitar-que-existe-relauxe7uxe3o-linear-entre-as-variuxe1veis.}{%
\paragraph{O pvalor (0.0000000000344) \textless{} 0,10 indica que
devemos rejeitar a hipótese nula, ou seja, aceitar que existe relação
linear entre as
variáveis.}\label{o-pvalor-0.0000000000344-010-indica-que-devemos-rejeitar-a-hipuxf3tese-nula-ou-seja-aceitar-que-existe-relauxe7uxe3o-linear-entre-as-variuxe1veis.}}

\hypertarget{d-interprete-os-paruxe2metros-do-modelo-e-o-coeficiente-de-determinauxe7uxe3o.}{%
\subsubsection{d) Interprete os parâmetros do modelo e o coeficiente de
determinação.}\label{d-interprete-os-paruxe2metros-do-modelo-e-o-coeficiente-de-determinauxe7uxe3o.}}

\hypertarget{ux3b21-rejeita-a-hipuxf3tese-nula}{%
\paragraph{β1: Rejeita a hipótese
Nula}\label{ux3b21-rejeita-a-hipuxf3tese-nula}}

\hypertarget{ux3b20-rejeita-a-hipuxf3tese-nula}{%
\paragraph{β0: Rejeita a hipótese
Nula}\label{ux3b20-rejeita-a-hipuxf3tese-nula}}

\hypertarget{r2-uxe9-07288-7288-dos-saluxe1rios-dos-candidatos-suxe3o-explicados-pelas-suas-notas-nos-processos-seletivos.}{%
\paragraph{R2 é 0,7288: 72,88\% dos salários dos candidatos são
explicados pelas suas notas nos processos
seletivos.}\label{r2-uxe9-07288-7288-dos-saluxe1rios-dos-candidatos-suxe3o-explicados-pelas-suas-notas-nos-processos-seletivos.}}

\hypertarget{se-um-determinado-funcionuxe1rio-tiver-hipoteticamente-nota-0-seu-saluxe1rio-seruxe1-de-r-2.78888}{%
\paragraph{Se um determinado funcionário tiver hipoteticamente nota 0,
seu salário será de R\$
2.788,88}\label{se-um-determinado-funcionuxe1rio-tiver-hipoteticamente-nota-0-seu-saluxe1rio-seruxe1-de-r-2.78888}}

\hypertarget{e-apresente-a-equauxe7uxe3o-do-modelo-estimada.}{%
\subsubsection{e) Apresente a equação do modelo
estimada.}\label{e-apresente-a-equauxe7uxe3o-do-modelo-estimada.}}

y\_estimado \textless- 2788.88 + 104.04 * x

\hypertarget{f-estime-o-valor-do-saluxe1rio-para-um-candidato-que-possui-a-nota-muxe9dia-igual-a-7}{%
\subsubsection{f) Estime o valor do salário para um candidato que possui
a nota média igual a
7}\label{f-estime-o-valor-do-saluxe1rio-para-um-candidato-que-possui-a-nota-muxe9dia-igual-a-7}}

O valor estimado do salário para quem tirou nota 7 na prova é R\$
3517.16

\end{document}
